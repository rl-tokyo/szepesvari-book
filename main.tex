\documentclass{jsarticle}
\usepackage[fleqn]{amsmath}
\usepackage{graphicx}

% commands
\newcommand{\argmin}{\mathop{\rm argmin}}
\newcommand{\argmax}{\mathop{\rm argmax}}
\newcommand{\States}{\mathcal{X}}
\newcommand{\Actions}{\mathcal{A}}
\newcommand{\st}{x}
\newcommand{\St}{X}
\newcommand{\action}{a}
\newcommand{\nextaction}{a'}
\newcommand{\Action}{A}
\newcommand{\Nextaction}{A'}
\newcommand{\reward}{r}
\newcommand{\Reward}{R}
\newcommand{\MDP}{\mathcal{M}}
\newcommand{\nextstate}{y}
\newcommand{\Nextstate}{Y}
\newcommand{\Var}[1]{{\mathrm{Var}}\left[#1\right]}

\date{}
\title{szepesvari本: 数式補足}

\begin{document}
\maketitle

\section*{2.1.2 逐一訪問モンテカルロ法}

\paragraph{$\Var{\hat{V}_t(3)} \approx 1/(10\,k)$?}
状態$3$から状態$4$への遷移が$n = 10k$回観測されたとすると,
価値関数はベルヌーイ分布に従う確率変数$X_1, \ldots, X_n$を使って
$\hat{V}_t(3) = \sum_{k=1}^n X_k / n$と推定できる.
ここで,$Y = \sum_{k=1}^n X_k$は$n$回コインを投げて何回表が出るかに対応するので,
二項分布に従う.
二項分布の分散は$n\,p(1-p)$なので,
\begin{equation}
\Var{\hat{V}_t(3)} = n \cdot \Var{Y} = \frac{p(1-p)}{n}
\end{equation}
となる.

\section*{2.4 Dynamic programming algorithms for solving MDPs}

\subsection*{$Q^{k}$に対してグリーディな方策に関するバウンドの証明}
\begin{equation}
V^\pi(\st) \ge V^*(\st) - \frac{2}{1-\gamma} \, \|Q-Q^*\|_{\infty}
\end{equation}

\end{document}
